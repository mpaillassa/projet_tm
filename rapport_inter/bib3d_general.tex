Dans cette partie, nous allons présenter différentes bibliothèques graphiques 3D que nous pourrions utiliser: OpenSceneGraph, Ogre, OpenFrameworks et irrlicht qui utilisent OpenGL et Blender, Unity, Qt3D, Babylon.js et Three.js. Nous n'avons pas voulu utiliser uniquement OpenGL car cela aurait demandé plus de code et de temps pour un résultat équivalent. \\
Toutes ces biliothèques sont déployables sur pratiquement toutes les plateformes (y compris smartphones et tablettes). Le fait que les premières soient orientées objets permet d'avoir une programmation plus haut niveau tout en profitant des performances d'OpenGL.  Elles intègrent également des packages additionnels pour faire des rendus 3D plus poussés. \\
Blender et Unity sont plus orientées sur de l'animation et du jeu vidéo donc sur du rendu 3D. \\
Babylon.js et Three.js se déploient sur navigateur via javascript et utilisent WebGL.

- vérifier la version de C++ utilisée \\
- communautés dév (bibliothèque maintenue et contribution des utilisateurs) et utilisateurs (domaines d'applications variés) \\
