\subsection{Robot Art}
Dans cet article, les auteurs proposent un système d'assemblage de robot dans un environnement virtuel en 3D. Le but est ensuite de pouvoir les mettre en mouvement. Chaque robot est composé de différentes parties organisées hiérarchiquement. Pour faire bouger les différentes parties des robots assemblés, le module de visualisation 3D proposé se base sur des matrices de transformation (translation, rotation, facteur d'échelle). Quand une partie subit une transformation, celle-ci est appliquée à chacune des parties filles. \\
Une interface permet à l'utilisateur d'assembler les robots et de les observer depuis quatre points de vue différents. 