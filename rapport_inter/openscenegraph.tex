Dans cette partie, nous allons présenter les bibliothèques graphiques 3D que nous pourrions utiliser: OpenSceneGraph, Ogre, OpenFrameworks, irrlicht \\
Toutes ces biliothèques utilisent OpenGL et sont déployables sur pratiquement toutes les plateformes (y compris smartphones et tablettes). Le fait qu'elles soient orientées objets permet d'avoir une programmation plus haut niveau tout en profitant des performances d'openGL. Elles intègrent également des packages additionnels pour faire des rendus 3D plus poussés.

- vérifier la version de C++ utilisée

- OO: programmation plus haut niveau et bonne performances

- Ogre et Osg: graphes
- Ogre: plus tournée jeux vidéos 
- openFrameworks: beaucoup de dépendances

