\subsection{i-score}

Tout le travail réalisé devra s'interfacer avec le logiciel i-score qui sera utilisé pour écrire les chorégraphies. Pour communiquer avec un autre logiciel, i-score intègre le format OSC qui utilise le protocole UDP. Un message OSC se compose comme ceci:
\begin{lstlisting}
<address pattern> <data type> <data>
\end{lstlisting}
Address pattern est l'URL désignant la destination du message, et data type une chaîne de caractères indiquant le type de données contenues dans data.
Cependant, le protocole Minuit peut aussi être utilisé: ce protocole s'ajoute au protocole OSC, et permet de mieux organiser les données transmises. Les adresses OSC sont organisées dans un arbre et le logiciel pourra demander via le protocole Minuit à i-score de lui communiquer la valeur de tel ou tel paramètre.  

