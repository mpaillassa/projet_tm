Pour chaque bib : 
- particularité / explication brève
- domaine d'utilisation
- moteur physique 
- philosophie
- license
- communauté dév et utilisateurs
- stabilité
- desavantages ?

\subsection{OpenSceneGraph}
- explication : open source/C++/
philosophie/choix :  manière différente de visualisation -> difficulté à prendre en main (désavantage pour nous et pour les suivants) > difficulté de maintenabilité
- communauté : peu active

\subsection{OpenFrameworks}
- explication : open source/C++/

%- fonctionne avec OpenGL, OpenCV (traitement image pour trajectoires?)
- philosophie : "it should be collaborative, usable and simple, consistent and intuitive, cross-platform, powerful, and extensible. openFrameworks is also driven by a "do it with others" (DIWO) philosophy"
- désavantage installation difficile ?


\subsection{Ogre}
- explication : open source/C++/
domaine d'utilisation :  plus tourné jeux vidéos 
license : GNU LGPL \\

\subsection{Irrlicht}
- explication : open source/C++/
moteur physique : oui avec détection de collision
stabilité : pas très stable 

\subsection{Blender et Unity}
- explication : Blender : open source
				Unity : C\# / pas open source
				
- communauté  : utilisateur active
- domaine d'utilisation : tourné jeux vidéos mais surtout modélisation et pas vraiment visualisation
--> peut être trop de fonctionnalités liées à la modélisation qui nous seraient inutiles...?
- désavantage pour Unity -> pas open source

\subsection{Qt3D}
- explication : open source/ C++
moteur physique : détection de collision
stabilité : pas vraiment stable = en cours de développement

\subsection{Babylon.js et Three.js}
- explication: moteur 3D javascript utilisant webgl, permet d'avoir un rendu visuel dans le navigateur
- désavantage : 
		besoin de mettre dans navigateur  ? 
	javascript : pas adapté pour la suite du projet, a priori le projet dure plus que nos trois mois (environ un an ou deux) donc si on considère le fait que la visualisation devienne sophistiquée (par ex : possibilité de modifier la chorégraphie en elle-même ? pas seulement visualisation mais aussi création) le langage javascript ne semble pas adapté...
		GROS NEGATIF : difficulté à récupérer les données d'iscore via UDP avec Javascript? (utiliser node.js ?)