\subsection{OpenSceneGraph}
openscenegraph -> manière différente de visualisation = difficulté à prendre en main (désavantage pour nous et pour les suivants) > difficulté de maintenabilité

\subsection{OpenFrameworks}
- openframeworks -> installation difficile ?
- C++
- open source
- fonctionne avec OpenGL, OpenCV (traitement image pour trajectoires?)
- philisophie : 
OF:  it should be collaborative, usable and simple, consistent and intuitive, cross-platform, powerful, and extensible. openFrameworks is also driven by a "do it with others" (DIWO) philosophy. 

\subsection{Ogre}
ogre : - plus tourné jeux vidéos 

\subsection{Irrlicht}
irrlicht: + détection de collision, - pas très stable 



\subsection{Blender et Unity}

Blender / unity -> trop lourd ? trop de trucs inutiles 
				-> pas besoin de modélisation
				-> plus pour la création que la visualisation, mais à titre comparatif mettre Unity dans le tableau ?\\

\subsection{Qt3D}
Qt3D -> en cours de développement

\subsection{Babylon.js et Three.js}

Babylon.js / Three.js -> --> moteur 3D javascript utilisant webgl
		besoin de mettre dans navigateur  ? 
	javascript : pas adapté pour la suite du projet, a priori le projet dure plus que nos trois mois (environ un an ou deux) donc si on considère le fait que la visualisation devienne sophistiquée (par ex : possibilité de modifier la chorégraphie en elle-même ? pas seulement visualisation mais aussi création) le langage javascript ne semble pas adapté...
		GROS NEGATIF : difficulté à récupérer les données d'iscore via UDP avec Javascript? (utiliser node.js ?)