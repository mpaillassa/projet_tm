intro de cette partie :

Dans cette partie, un bilan de l'étude des différents biblitohèques sera présentée sous forme de tableau avec cinq critères jugés importants dans le choix de la biliothèque. Parmi ces cinq, ne seront pas présentées ceux élémentaires telles que la présence d'un moteur 3D complet, l'aspect multi-plateforme, pour lesquelles il semble évident que toutes les bibliothèques citées jusque là les remplissent.
%\begin{itemize}
%%\item taille adaptée au projet (pas de surplus de fonctionnalités)
%\item base 3D (gestion caméra, scène)
%\item multi-plateforme (tablette aussi)
%\end{itemize}
Certains critères choisis pour ce bilan ont été réfléchi en vue de l'interopérabilité de notre logiciel avec i-score : l'aspect open source (notamment license GPL) et le langage utilsé. En effet, il nous semblait plus pertinant de respecter une unicité de langage avec i-score, et donc choisir le C++. D'autres critères 

Les critères qui ont donc été choisi pour ce bilan sont : 
\begin{itemize}
\item open source et notamment license GPL pour 
\item détection de collisions (moteur physique)
\item auto-suffisant (facilité d'utilisation d'UDP)
\item unicité du langage avec i-score (C++)
\item maintenabilité/stabilité  (facilité à prendre en main et à maintenir (stable))
\item maintenabilité (bonne communauté active)
\end{itemize} 
certains critères ne seront pas traités, tellement ces derniers sont élémentaires :
notamment : 


\newpage
\begin{landscape}
\hspace{-2cm} 
\begin{tabular}{l|c|c|c|c|c|c|c}
Bibliothèque & open source & moteur physique & auto-suffisant & langage & maintenabilité & stabilité & communauté active\\
\hline
OpenFrameworks & Oui & & &  & & \\
OpenSceneGraph & Oui & & &  & & \\ 
OGRE & Oui & &  & & &\\
Qt3D & Oui & &  & & &\\
Unity & Non &  & & & &\\
Babylon.js  & Non & & & & & \\
\end{tabular}
\end{landscape}
