Il existe déjà des logiciels permettant de réaliser un résultat semblable à ce qui est attendu. Il y a notamment le logiciel StageViz pour la visualisation en 3D en temps réel une performance scénique, qui peut être utilisé avec des outils de conception de performance scénique. Nous retrouvons ici, un cas très similaire à notre objectif avec la présence de visualisation mais aussi d'interopérabilité avec un outil de conception scénique.

Les auteurs de StageViz expliquent qu'à la base, un sondage avait été réalisé auprès des différentes personnes du milieu de la performance artistique (artistes, ingénieurs, metteur en scène) pour cibler les besoins d'un outil de visualisation. Trois éléments principaux ont été retenus : les objets, la manipulation de la scène et la chronologie. En effet, la présence des différents objets 3D, les décors mais aussi les acteurs, permettent un rendu réaliste de la scène. Concernant la manipulation de la scène, c'est ce qui représente l'avantage d'avoir une modélisation virtuelle : la possibilité de bouger la camera dans toute la scène 3D pour avoir un aperçu sous tous les angles. La chronologie permet d'avoir une évoution des états des objets 3D dans la scène.

Contrairement à StageViz dont le but semble être de simuler de façon la plus réaliste possible une performance scénique, le but de notre projet repose plus sur une visualisation pour aider l'écriture de la chorégraphie. L'idée est donc d'avoir un rendu réaliste mais surtout permettre une vision simplifiée et globale de la chorégraphie avec tous les robots.

%- traite sur les performances, 
%	parle de "storyboarding" -> notion de temps 
%	
%- visualisation de la performance avec des composantes créées par l'utilisateur
%BUT LOGICIEL "StageViz provides visualized scene image
%	based on the timeline and virtual scene components created by
%	a user." mais dans les articles précédents, expliqué que le logociel permet aussi de contrôler différents outils liés à la mise en scène (SFX, lights...)
%
%- BUT ARTICLE :  établir une connection entre la visualisation et la performance réelle sur scène
%
%- pourquoi ce logiciel -> sondage auprès des gens du milieu de la performance artistique (ingénieurs, artistes, metteur en scène)
%	trois critères ont été relevés et retenus
%		objet / "timeline"~chronologie / manipulation de la scène 
%
%- logiciel basé sur Direct3D pour faire le rendu 3D
%
%- expérience utilisant des costumes lumineux 
%	  résultat : le rendu 3D permet de faire une modélisation très fine de la réalité 
%		  montre que ce logiciel peut être utilisé pour créer/réaliser une mise en scène puisque le rendu visuel généré par ce logiciel est très proche de ce qui peut être obtenu en vrai.