Le but du projet Rain of music est de pouvoir scénariser un spectacle impliquant des robots et de la musique. Deux types de robots sont envisagés: des métabots (robots terrestres) et des drones (robots volant sans pilote). Ces robots réaliseraient une chorégraphie dans un espace donné et embarqueraient des haut parleurs pour émettre des sons. Ce projet pluridisplinaire implique trois parties: des étudiants en arts pour l'écriture de la chorégraphie, des étudiants roboticiens pour gérer les problématiques de communication et de localisation des robots, et des étudiants en technologies multimédia (nous-mêmes) pour les sons et l'interface qui permettra aux artistes d'écrire et de simuler des chorégraphies.

Notre travail s'inscrit dans le tout début du projet (3 premiers mois) qui durera bien plus longtemps pour arriver aux objectifs finaux cités précédemment. Dans ce contexte, nous avons dû adapter nos objectifs. Ainsi, nous nous sommes fixés de réaliser un logiciel qui permettra de simuler une chorégraphie dans un espace 3D pour aider les artistes dans l'écriture d'une chorégraphie. Dans cet état de l'art, nous allons donc présenter des techniques existantes de visualisation et de simulation, puis nous nous intéresserons aux bibliothèques 3D que nous pourrions utiliser.