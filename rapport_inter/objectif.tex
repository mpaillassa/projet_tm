Avant de nous intéresser plus en détail à ce que nous pouvons utiliser pour atteindre nos objectifs, nous allons préciser ceux-ci.

Comme dit précédemment, notre travail consistera à écrire un logiciel permettant de visualiser et de simuler une chorégraphie écrite par les artistes. Cette chorégraphie est écrite grâce au logiciel i-score, qui est un séquenceur: il permet d'écrire des scénarios en programmant des évènements dans le temps. Notre travail impliquera donc deux problématiques: 
\begin{itemize}
\item la récupération des données d'i-score dans le logiciel, de manière à avoir les positions des robots.
\item l'affichage des robots dans une scène 3D, qui impliquera d'autres problématiques comme la gestion de collisions.
\end{itemize} 

