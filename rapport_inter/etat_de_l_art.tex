%%%%%%%%%%%%%%%%%%%%%%%%%%%%%%%%%%%%%%%%%%%%%%%%%%%%%%%%%%%%%%%%%%%%%%
% LaTeX Example: Project Report
%
% Source: http://www.howtotex.com
%
% Feel free to distribute this example, but please keep the referral
% to howtotex.com
% Date: March 2011 
% 
%%%%%%%%%%%%%%%%%%%%%%%%%%%%%%%%%%%%%%%%%%%%%%%%%%%%%%%%%%%%%%%%%%%%%%

% Edit the title below to update the display in My Documents
%\title{Project Report}
%
%%% Preamble
\documentclass[paper=a4,12pt]{article}
\usepackage[T1]{fontenc}
\usepackage{kpfonts}
%\usepackage[fontsize=12pt]{scrextend}
\usepackage[utf8]{inputenc}
\usepackage[french]{babel}

% English language/hyphenation

\usepackage{amsmath,amsfonts,amsthm} % Math packages
%\usepackage[pdftex]{graphicx}	
\usepackage{url}
\usepackage[bottom=10em]{geometry}
\usepackage{float}
\usepackage{xcolor}
\usepackage{enumitem}
\usepackage{rotating}

%%% Custom sectioning
%\usepackage{sectsty}
%\allsectionsfont{\normalfont\scshape}

%% Language definition package (for XML Annexe)
\usepackage{listings}
\usepackage{color}

%% Local modification of margins
%\newenvironment{changemargin}[2]{\begin{list}{}{%
%      \setlength{\topsep}{0pt}%
%      \setlength{\leftmargin}{0pt}%
%      \setlength{\rightmargin}{0pt}%
%      \setlength{\listparindent}{\parindent}%
%      \setlength{\itemindent}{\parindent}%
%      \setlength{\parsep}{0pt plus 1pt}%
%      \addtolength{\leftmargin}{#1}%
%      \addtolength{\rightmargin}{#2}%
%    }\item }{\end{list}}
%%%

%%% Custom headers/footers (fancyhdr package)
%\usepackage{fancyhdr}
%\pagestyle{fancyplain}
%\fancyhead{}											% No page header
%\fancyfoot[L]{}											% Empty 
%\fancyfoot[C]{}											% Empty
%\fancyfoot[C]{\thepage}									% Pagenumbering
%\renewcommand{\headrulewidth}{0pt}			% Remove header underlines
%\renewcommand{\footrulewidth}{0pt}				% Remove footer underlines
%\setlength{\headheight}{13.6pt}


%%% Equation and float numbering
\numberwithin{equation}{section}		% Equationnumbering: section.eq#
\numberwithin{figure}{section}			% Figurenumbering: section.fig#
\numberwithin{table}{section}				% Tablenumbering: section.tab#

%Graphics path
%\graphicspath{./Images/}

%%% Maketitle metadata
\newcommand{\horrule}[1]{\rule{\linewidth}{#1}} 	% Horizontal rule

\title{
  %\vspace{-1in} 			
  %\usefont{OT1}{bch}{b}{n}
  \horrule{1.5pt} \\[0.5cm]	
  \Huge \textbf{Projet Rain Of Music : \\ État de l'art} \\ [20pt]
 % \huge TM, \\ Université d'Osaka \\ [15pt]
 	%	\vspace{1cm}  
    \LARGE Année scolaire 2015-2016 \\ 
  \horrule{1.5pt} \\[0.5cm]
  %
}

\author{						
    \LARGE \underline{Encadrants} : Jean-Michaël Celerier, \\
   					\LARGE	\hspace{5cm} Myriam De Sainte-Catherine\\			
   	\vspace{1cm} 
   	\normalfont
   	\LARGE Élèves :  Akané LEVY, Maxime PAILLASSA    
}
\date{}

%%% Begin document
\begin{document}
\graphicspath{{./imgs/}{.}}
\maketitle

\begin{figure}[H]
  \centering\includegraphics[scale=1.2]{logo_enseirb.png}
\end{figure}

\newpage

\tableofcontents

\newpage
\normalsize
%\begin{changemargin}{-1cm}{-1cm}

\section{Introduction}
présenter les différents acteurs
Le but du projet Rain of music est de pouvoir scénariser un spectacle impliquant des robots et de la musique. Deux types de robots sont envisagés: des métabots (robots terrestres) et des drones (robots volant sans pilote). Ces robots réaliseraient une chorégraphie dans un espace donné et embarqueraient des haut parleurs pour émettre des sons. Ce projet pluridisplinaire implique trois parties: des étudiants en arts pour l'écriture de la chorégraphie, des étudiants roboticiens pour gérer les problématiques de communication et de localisation des robots, et des étudiants en technologies multimédia (nous-mêmes) pour les sons et l'interface qui permettra aux artistes d'écrire et de simuler des chorégraphies.

Notre travail s'inscrit dans le tout début du projet (3 premiers mois) qui durera bien plus longtemps pour arriver aux objectifs finaux cités précédemment. Dans ce contexte, nous avons dû adapter nos objectifs. Ainsi, nous nous sommes fixés de réaliser un logiciel qui permettra de simuler une chorégraphie dans un espace 3D pour aider les artistes dans l'écriture d'une chorégraphie. Dans cet état de l'art, nous allons donc présenter des techniques existantes de visualisation et de simulation, puis nous nous intéresserons aux bibliothèques 3D que nous pourrions utiliser.

\section{Objectifs}
explication du résultat à obtenir

- visualisation en 3D de la chorégraphie des robots, à l'aide des données envoyés par i-score 

\section{Existant}
présentation iscore (gestion d'évènements)

Il existe déjà des logiciels permettant de réaliser un résultat semblable à ce qui est attendu. Il y a notamment le logiciel StageViz pour la visualisation en 3D en temps réel une performance scénique, qui peut être utilisé avec des outils de conception de performance scénique. Nous retrouvons ici, un cas très similaire à notre objectif avec la présence de visualisation mais aussi d'interopérabilité avec un outil de conception scénique.

Les auteurs de StageViz expliquent qu'à la base, un sondage avait été réalisé auprès des différentes personnes du milieu de la performance artistique (artistes, ingénieurs, metteur en scène) pour cibler les besoins d'un outil de visualisation. Trois éléments principaux ont été retenus : les objets, la manipulation de la scène et la chronologie. En effet, la présence des différents objets 3D, les décors mais aussi les acteurs, permettent un rendu réaliste de la scène. Concernant la manipulation de la scène, c'est ce qui représente l'avantage d'avoir une modélisation virtuelle : la possibilité de bouger la camera dans toute la scène 3D pour avoir un aperçu sous tous les angles. Les deux éléments précédant constituant la réalisation visuelle spatiale, il reste l'élément temporel qu'est la chronologie, qui a été rendu possible grâce à la visualisation en temps réel.

À partir de cet exemple, deux besoins qui devront être satisfaits par notre réalisation sont apparues : le critère spatiale avec une gestion de la scène 3D composée de ses robots et le critère temporel représenté par l'évolution des positions des robots dans la scène.


%- traite sur les performances, 
%	parle de "storyboarding" -> notion de temps 
%	
%- visualisation de la performance avec des composantes créées par l'utilisateur
%BUT LOGICIEL "StageViz provides visualized scene image
%	based on the timeline and virtual scene components created by
%	a user." mais dans les articles précédents, expliqué que le logociel permet aussi de contrôler différents outils liés à la mise en scène (SFX, lights...)
%
%- BUT ARTICLE :  établir une connection entre la visualisation et la performance réelle sur scène
%
%- pourquoi ce logiciel -> sondage auprès des gens du milieu de la performance artistique (ingénieurs, artistes, metteur en scène)
%	trois critères ont été relevés et retenus
%		objet / "timeline"~chronologie / manipulation de la scène 
%
%- logiciel basé sur Direct3D pour faire le rendu 3D
%
%- expérience utilisant des costumes lumineux 
%	  résultat : le rendu 3D permet de faire une modélisation très fine de la réalité 
%		  montre que ce logiciel peut être utilisé pour créer/réaliser une mise en scène puisque le rendu visuel généré par ce logiciel est très proche de ce qui peut être obtenu en vrai.
Dans cet article, les auteurs proposent un système d'assemblage de robot dans un environnement virtuel en 3D. Le but est ensuite de pouvoir les mettre en mouvement. Chaque robot est composé de différentes parties organisées hiérarchiquement. Pour faire bouger les différentes parties des robots assemblés, le module de visualisation 3D proposé se base sur des matrices de transformation (translation, rotation, facteur d'échelle). Quand une partie subit une transformation, celle-ci est appliquée à chacune des parties filles. \\
Une interface permet à l'utilisateur d'assembler les robots et de les observer depuis quatre points de vue différents. 
articles, comparer par rapport à nos objectifs \\
justifie notre recherche de bibliothèques



\section{Ce qu'on peut utiliser}
Qt3D -> en cours de développement

OGRE -> ...? 

openframeworks -> installation difficile ?

openscenegraph -> manière différente de visualisation = difficulté à prendre en main (désavantage pour nous et pour les suivants) > difficulté de maintenabilité

opengl -> "trop" de travail pour peu de temps  = avec un langage plus haut niveau, il serait possible de coder en moins de temps pour un résultat équivalent

Blender / unity -> trop lourd ? trop de trucs inutiles 
				-> pas besoin de modélisation


besoin de mettre dans navigateur  ? --> moteur 3D javascript utilisant webgl
	javascript : pas adapté pour la suite du projet, a priori le projet dure plus que nos trois mois (environ un an ou deux) donc si on considère le fait que la visualisation devienne sophistiquée (par ex : possibilité de modifier la chorégraphie en elle-même ? pas seulement visualisation mais aussi création) le langage javascript ne semble pas adapté...
		GROS NEGATIF : difficulté à récupérer les données d'iscore via UDP avec Javascript? (utiliser node.js ?)
		
\input{openframeworks.tex}
\subsection{OpenSceneGraph}

OpenSceneGraph est une bibliothèque opensource utilisée dans beaucoup de domaines comme la simulation, la réalité augmentée ou virtuelle, la visualisation scientifique, les jeux, l'éducation. Ecrite en C++ et OpenGL, elle peut être déployée sur pratiquement toutes les plateformes (y compris des smartphones ou tablettes) et est compatible avec toutes les versions d'OpenGL de 1.0 à 4.2. \\
OpenSceneGraph encapsule les fonctionnalités d'OpenGL et offre donc une programmation plus haut niveau tout en gardant des bonnes performances. De plus, sa modularité permet d'utiliser  \\
Un monde 3D est représenté par un graphe de noeuds regroupés logiquement et spatialement en sous-graphes.  
bibliothèques, moteurs 3D \\
tableau comparatif selon les critères (taille de la bib)

\section{choix final}
intro de cette partie :

Dans cette partie, un bilan de l'étude des différents biblitohèques sera présentée sous forme de tableau avec cinq critères jugés importants dans le choix de la biliothèque. Parmi ces cinq, ne seront pas présentées ceux élémentaires telles que la présence d'un moteur 3D complet, l'aspect multi-plateforme, pour lesquelles il semble évident que toutes les bibliothèques citées jusque là les remplissent.
%\begin{itemize}
%%\item taille adaptée au projet (pas de surplus de fonctionnalités)
%\item base 3D (gestion caméra, scène)
%\item multi-plateforme (tablette aussi)
%\end{itemize}
Certains critères choisis pour ce bilan ont été réfléchi en vue de l'interopérabilité de notre logiciel avec i-score : l'aspect open source (notamment license GPL) et le langage utilsé. En effet, il nous semblait plus pertinant de respecter une unicité de langage avec i-score, et donc choisir le C++. D'autres critères 

Les critères qui ont donc été choisi pour ce bilan sont : 
\begin{itemize}
\item open source et notamment license GPL pour 
\item détection de collisions (moteur physique)
\item auto-suffisant (facilité d'utilisation d'UDP)
\item unicité du langage avec i-score (C++)
\item maintenabilité/stabilité  (facilité à prendre en main et à maintenir (stable))
\item maintenabilité (bonne communauté active)
\end{itemize} 
certains critères ne seront pas traités, tellement ces derniers sont élémentaires :
notamment : 


\newpage
\begin{landscape}
\hspace{-2cm} 
\begin{tabular}{l|c|c|c|c|c|c|c}
Bibliothèque & open source & moteur physique & auto-suffisant & langage & maintenabilité & stabilité & communauté active\\
\hline
OpenFrameworks & Oui & & &  & & \\
OpenSceneGraph & Oui & & &  & & \\ 
OGRE & Oui & &  & & &\\
Qt3D & Oui & &  & & &\\
Unity & Non &  & & & &\\
Babylon.js  & Non & & & & & \\
\end{tabular}
\end{landscape}

justification

\section{Conclusion}
idée de notre réalisation concrètement
- re-justification de la bibliothèque choisie en reprenant le bilan précédent
- expliquer commment on compte utiliser l'outil choisi pour notre projet


  \newpage

%  \bibliographystyle{plain}
%  \bibliography{bibli.bib}
%  \newpage
%
  \appendix
  \part*{Annexes}
  \input{annexe.tex}
 %   \section{Autres}% \label{annexe:modeles}
    %\input{notice_installation.tex} 
  

  
 % \newpage 
  

 

%\end{changemargin}

%%% End document
\end{document}
